\documentclass[a4paper,12pt]{article}
\usepackage{amssymb} % needed for math
\usepackage{amsmath} % needed for math
\usepackage[utf8]{inputenc} % this is needed for german umlauts
\usepackage[ngerman]{babel} % this is needed for german umlauts
\usepackage[T1]{fontenc}    % this is needed for correct output of umlauts in pdf
\usepackage[margin=2.5cm]{geometry} %layout
\usepackage{booktabs}

% this is needed for forms and links within the text
\usepackage{hyperref}  

% glossar, see http://en.wikibooks.org/wiki/LaTeX/Glossary
% has to be loaded AFTER hyperref so that entries are clickable
\usepackage[nonumberlist]{glossaries} 

% The following is needed in order to make the code compatible
% with both latex/dvips and pdflatex.
\ifx\pdftexversion\undefined
\usepackage[dvips]{graphicx}
\else
\usepackage[pdftex]{graphicx}
\DeclareGraphicsRule{*}{mps}{*}{}
\fi

\makeglossary 

%%%%%%%%%%%%%%%%%%%%%%%%%%%%%%%%%%%%%%%%%%%%%%%%%%%%%%%%%%%%%%%%%%%%%%
% Variablen                                 						 %
%%%%%%%%%%%%%%%%%%%%%%%%%%%%%%%%%%%%%%%%%%%%%%%%%%%%%%%%%%%%%%%%%%%%%%
\newcommand{\authorName}{Lena Gregor, Dominik Horn}
\newcommand{\auftraggeber}{Lehrstuhl für Datenbanksystem}
\newcommand{\auftragnehmer}{Database Systems Studenten Team}
\newcommand{\projektName}{Wahlinformationssystem}
\newcommand{\tags}{\authorName, Lastenheft, TUM, Universität Augsburg, LMU}
\newcommand{\glossarName}{Glossar}
\title{\projektName~(Lastenheft)}
\author{\authorName}
\date{\today}

%%%%%%%%%%%%%%%%%%%%%%%%%%%%%%%%%%%%%%%%%%%%%%%%%%%%%%%%%%%%%%%%%%%%%%
% PDF Meta information                                 				 %
%%%%%%%%%%%%%%%%%%%%%%%%%%%%%%%%%%%%%%%%%%%%%%%%%%%%%%%%%%%%%%%%%%%%%%
\hypersetup{
  pdfauthor   = {\authorName},
  pdfkeywords = {\tags},
  pdftitle    = {\projektName~(Lastenheft)}
} 
 
%%%%%%%%%%%%%%%%%%%%%%%%%%%%%%%%%%%%%%%%%%%%%%%%%%%%%%%%%%%%%%%%%%%%%%
% Create a shorter version for tables. DO NOT CHANGE               	 %
%%%%%%%%%%%%%%%%%%%%%%%%%%%%%%%%%%%%%%%%%%%%%%%%%%%%%%%%%%%%%%%%%%%%%%
\newcommand\addrow[2]{#1 &#2\\ }

\newcommand\addheading[2]{#1 &#2\\ \hline}
\newcommand\tabularhead{\begin{tabular}{lp{13cm}}
\hline
}

\newcommand\addmulrow[2]{ \begin{minipage}[t][][t]{2.5cm}#1\end{minipage}% 
   &\begin{minipage}[t][][t]{8cm}
    \begin{enumerate} #2   \end{enumerate}
    \end{minipage}\\ }

\newenvironment{usecase}{\tabularhead}
{\hline\end{tabular}}




%%%%%%%%%%%%%%%%%%%%%%%%%%%%%%%%%%%%%%%%%%%%%%%%%%%%%%%%%%%%%%%%%%%%%%
% THE DOCUMENT BEGINS             	                              	 %
%%%%%%%%%%%%%%%%%%%%%%%%%%%%%%%%%%%%%%%%%%%%%%%%%%%%%%%%%%%%%%%%%%%%%%
\begin{document}
 \pagenumbering{roman}
 \begin{titlepage}
\thispagestyle{empty} % no page number
\centering

\vspace*{4cm}
{\huge\bfseries Pflichtenheft\par}
\vspace{1cm}
{\huge\bfseries \projektName\par}
\vspace{0.5cm}
{\large \subtitle\par}
\vspace{1cm}
{\large\itshape \authorName\par}

\vfill

%\renewcommand{\arraystretch}{1.5}
\begin{tabular}{|a|l|}
	\hline
	\textcolor{white}{\textbf{Projekt}} & \projektName \\
	\hline
	\textcolor{white}{\textbf{Auftraggeber}} & \auftraggeber \\
	\hline
	\textcolor{white}{\textbf{Auftragnehmer}} & \auftragnehmer \\
	\hline
	\textcolor{white}{\textbf{Datum}} & \today \\
	\hline
\end{tabular}
%\renewcommand{\arraystretch}{1}

\end{titlepage}
         % Deckblatt.tex laden und einfügen
 \setcounter{page}{2}
 \tableofcontents          % Inhaltsverzeichnis ausgeben
 \clearpage
 \pagenumbering{arabic}
 
\section{Zielbestimmung}
%%%%%%%%%%%%%%%%%%%%%%%%%%%%%%%%%%%%%%%%%%%%%%%%%%%%%%%%%%%%%%%%%%%%%%
% Warum wird das Projekt gemacht?           						 %
%%%%%%%%%%%%%%%%%%%%%%%%%%%%%%%%%%%%%%%%%%%%%%%%%%%%%%%%%%%%%%%%%%%%%%
Der Freistaat Bayern möchte in Kooperation mit dem Lehrstuhl für Datenbanksysteme an der Technischen Universität München
eine digitales Wahlinformations- und Stimmabgabesystem von Studierenden des Elite Software Engineering Masterstudiengangs 
für Landtagswahlen aufbauen.
Das System soll dabei nicht nur die Ergebnisse für die Landtagswahlen 2013 und 2018 analysierbar machen (Sitzverteilung im
Landtag, Vergleiche zwischen Wahlen, Zusammensetzung des Landtags) sondern auch als sicheres Backendsystem für die
elektronische Stimmabgabe im Wahllokal dienen. Es müssen die gesetzlichen Regelungen und Normen und Datenschutzaspekte
berücksichtigt werden. Als gesetzliche Basis dienen die Regelungen zur Landtagswahl 2013.

\section{Produkteinsatz}
%%%%%%%%%%%%%%%%%%%%%%%%%%%%%%%%%%%%%%%%%%%%%%%%%%%%%%%%%%%%%%%%%%%%%%
% Wer ist die Zielgruppe?                   						 %
%%%%%%%%%%%%%%%%%%%%%%%%%%%%%%%%%%%%%%%%%%%%%%%%%%%%%%%%%%%%%%%%%%%%%%
Zielgruppe ist im Wesentlichen der Übungsleiter der Database Systems Vorlesung
(Christian Winter) und die Studierenden/Entwickler. Konzipiert und Entwickelt
werden soll die Anwendung jedoch zur Verwendung durch Wahlhelfer zur 
Stimmeneintragung für Bayrische Landtagswahlen, zur Stimmabgabe durch Wahlberechtigte
(ggf. nach externer Personalverifikation durch Wahlhelfer) und zur (statistischen)-Auswertung
von Wahlergebnissen durch den oder die WahlleiterIn sowie u.u interessierten BürgerInnen.

\section{Benutzerschnittstellen}
Die Informationsabfrage und (statistische) Auswertung von Wahlergebnissen erfolgt
in einer frei zugänglichen Weboberfläche. Hierbei soll auch der Vergleich zwischen verschiedenen
Wahlen möglich sein.
%
Das Eintragen von einer oder mehreren Stimmabgaben ist als backend Aufruf (Serverless Function)
für authentifizierte Nutzer zugänglich. Diese funktion machen sich Wahlhelfer oder Wahlleiter zunutze,
z.B. zum Übermitteln von Wahlergebnissen aus der jeweiligen Region.
%
Zur Einzelstimmabgabe im Wahllokal wird ebenfalls eine Weboberfläche zugänglich gemacht,
die erst nach Freigabe durch eine(n) WahlhelferIn zur Abgabe von maximal einer Erststimme
und einer Zweitstimme im jeweiligen Stimmkreis berechnet. Die vorangegangene Prüfung
durch den Wahlhelfer stellt sicher, dass die Person wahlberechtigt ist und nicht bereits
gewählt hat. Das System kann erweitert werden um den oder die WahlhelferIn beim
Verifizierungsschritt zu unterstützen.

\subsection{Auswertung von Wahlergebnissen}
\begin{center}
  \includegraphics{usecase.1}
\end{center}

\subsection{Eintragen von Wahlergebnissen durch Wahlhelfer}
\begin{center}
  \includegraphics{usecase.2}
\end{center}

\subsection{Stimmabgabe}
\begin{center}
  \includegraphics{usecase.3}
\end{center}

\section{Funktionale Anforderungen}
%%%%%%%%%%%%%%%%%%%%%%%%%%%%%%%%%%%%%%%%%%%%%%%%%%%%%%%%%%%%%%%%%%%%%%
% Was muss das Programm können?                   					 %
%%%%%%%%%%%%%%%%%%%%%%%%%%%%%%%%%%%%%%%%%%%%%%%%%%%%%%%%%%%%%%%%%%%%%%
\begin{usecase}
  \addheading{Nummer}{Beschreibung} 
  \addrow{/FA10/}{Lorem ipsum dolor sit amet, consetetur sadipscing elitr, sed diam}
  \addrow{/FA20/}{Lorem ipsum dolor sit amet, consetetur sadipscing elitr}
  \addrow{/FA30/}{adasdfasdf asdf asdfas}
\end{usecase}

\section{Produktdaten}
Verwaltete Produktdaten ergeben sich aus dem UML-Datenmodell.

\section{Nichtfunktionale Anforderungen}
\begin{usecase}
  \addheading{Nummer}{Beschreibung} 
  \addrow{/NF10/}{asdf asdf asdf asdf asdf asdf }
  \addrow{/NF20/}{ asdf asdf asdftergfgasdgewr asdfh}
\end{usecase}

\section{Abnahmekriterien}
Das System erlaubt nach importieren von realen Stimmdaten deren (statistische) Auswertung, d.h. 
insbesondere die Bestimmung der gewonnenen Mandate im bayrischen Landtag. Dabei müssen gesetzliche
Vorgaben akkurat abgebildet sein sowie Daten sicher, Datenschutzrechtlich korrekt und integer gehalten
werden. Des weiteren sollte eine graphisch Ansprechende Darstellung der (statistischen) Auswertung
einsehbar sein. 
Minimale Anforderung für den Einsatz im Wahllokal ist die Bereitstellung einer graphischen Schnittstelle zur Abgabe
von einzelnen Erst- und Zweitstimmen, welche unter Zurhilfenahme eines Wahlhelfers ausschließen können muss,
dass die Wahlgesetzgebung verletzt wird. Diese Anforderung ist insbesondere im gesetzlichen Rahmen zur
Stimmabgabe für bayrische Landtagswahlen zu verstehen.

\clearpage
%%%%%%%%%%%%%%%%%%%%%%%%%%%%%%%%%%%%%%%%%%%%%%%%%%%%%%%%%%%%%%%%%%%%%%
% Begriffslexikon zur Beschreibung des Produkts						 %
%%%%%%%%%%%%%%%%%%%%%%%%%%%%%%%%%%%%%%%%%%%%%%%%%%%%%%%%%%%%%%%%%%%%%%
%\newglossaryentry{sortierschluessel}
%{
%  name=Sortierschlüssel,
%  description={ein Schlüssel, anhand dessen diese Einträge sortiert werden}
%}
\newglossaryentry{Technologiestack}
{
  name=Technologiestack,
  description={Sammlung aller verwendeten Technologien im Projekt. Gegebenenfalls hierarchisch sortiert wenn aufeinander aufbauend}
}
\newglossaryentry{Client}
{
  name=Client,
  description={Programm, dass die Dienste eines Servers in Anspruch nimmt}
}
\newglossaryentry{Server}
{
  name=Server,
  description={Rechner, der für andere in einem Netzwerk mit ihm verbundene Systeme bestimmte Aufgaben übernimmt und von dem diese ganz oder teilweise abhängig sind}
}
\newglossaryentry{Datenbanksystem}
{
  name=Datenbanksystem,
  description={Ein Datenbanksystem (DBS) ist eine systematisch strukturierte, langfristig verfügbare Sammlung von Daten einschließlich der zur Verwaltung notwendigen Software}
}
\newglossaryentry{OLAP}
{
  name=OLAP,
  description={Online Analytical Processing}
}
\newglossaryentry{OLTP}
{
  name=OLTP,
  description={Online Transaction Processing}
}


% Setze den richtigen Namen für das Glossar
\renewcommand*{\glossaryname}{\section{\glossarName}}

% Drucke das gesamte Glossar
\glsaddall
\printglossaries

% Trage das Glossar in das Inhaltsverzeichnis ein
\stepcounter{section}
\addcontentsline{toc}{section}{\numberline {\thesection} \glossarName} 
\end{document}
