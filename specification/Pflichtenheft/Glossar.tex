%%%%%%%%%%%%%%%%%%%%%%%%%%%%%%%%%%%%%%%%%%%%%%%%%%%%%%%%%%%%%%%%%%%%%%
% Begriffslexikon zur Beschreibung des Produkts						 %
%%%%%%%%%%%%%%%%%%%%%%%%%%%%%%%%%%%%%%%%%%%%%%%%%%%%%%%%%%%%%%%%%%%%%%
%\newglossaryentry{sortierschluessel}
%{
%  name=Sortierschlüssel,
%  description={ein Schlüssel, anhand dessen diese Einträge sortiert werden}
%}
\newglossaryentry{Technologiestack}
{
  name=technologiestack,
  description={Sammlung aller verwendeten Technologien im Projekt. Gegebenenfalls hierarchisch sortiert wenn aufeinander aufbauend.}
}
\newglossaryentry{Client}
{
  name=client,
  description={Programm, dass die Dienste eines Servers in Anspruch nimmt}
}
\newglossaryentry{Server}
{
  name=server,
  description={Rechner, der für andere in einem Netzwerk mit ihm verbundene Systeme bestimmte Aufgaben übernimmt und von dem diese ganz oder teilweise abhängig sind}
}
\newglossaryentry{Datenbanksystem}
{
  name=datenbanksystem,
  description={Ein Datenbanksystem (DBS) ist eine systematisch strukturierte, langfristig verfügbare Sammlung von Daten einschließlich der zur Verwaltung notwendigen Software.}
}
\newglossaryentry{OLAP}
{
  name=olap,
  description={Online Analytical Processing.}
}
\newglossaryentry{OLTP}
{
  name=oltp,
  description={Online Transaction Processing.}
}


% Setze den richtigen Namen für das Glossar
\renewcommand*{\glossaryname}{\section{\glossarName}}

% Drucke das gesamte Glossar
\glsaddall
\printglossaries

% Trage das Glossar in das Inhaltsverzeichnis ein
\stepcounter{section}
\addcontentsline{toc}{section}{\numberline {\thesection} \glossarName}