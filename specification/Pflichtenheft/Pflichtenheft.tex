\documentclass[a4paper,12pt]{article}
%spellchecker
% !TeX spellcheck = de

\usepackage{amssymb} % needed for math
\usepackage{amsmath} % needed for math
\usepackage[utf8]{inputenc} % this is needed for german umlauts
\usepackage[ngerman]{babel} % this is needed for german umlauts
\usepackage[T1]{fontenc}    % this is needed for correct output of umlauts in pdf
\usepackage[margin=2.5cm]{geometry} %layout
\usepackage{booktabs}



% this is needed for forms and links within the text
\usepackage{hyperref}  

% Generate the glossary
\usepackage[nonumberlist]{glossaries}
\makeglossary

% The following is needed in order to make the code compatible
% with both latex/dvips and pdflatex.
\ifx\pdftexversion\undefined
\usepackage[dvips]{graphicx}
\else
\usepackage[pdftex]{graphicx}
\DeclareGraphicsRule{*}{mps}{*}{}
\fi

%%%%%%%%%%%%%%%%%%%%%%%%%%%%%%%%%%%%%%%%%%%%%%%%%%%%%%%%%%%%%%%%%%%%%%
% Variablen                                 						 %
%%%%%%%%%%%%%%%%%%%%%%%%%%%%%%%%%%%%%%%%%%%%%%%%%%%%%%%%%%%%%%%%%%%%%%
\newcommand{\authorName}{Lena Gregor, Dominik Horn}
\newcommand{\auftraggeber}{Lehrstuhl für Datenbanksysteme TUM}
\newcommand{\auftragnehmer}{\authorName}
\newcommand{\projektName}{Wahl und Informationssystem für bayrische Landtagswahlen}
\newcommand{\tags}{\authorName, Pflichtenheft, TUM, Universität Augsburg, LMU}
\newcommand{\subtitle}{Technische Universität München, Datenbanksysteme WS19/20}
\newcommand{\glossarName}{Glossar}

%%%%%%%%%%%%%%%%%%%%%%%%%%%%%%%%%%%%%%%%%%%%%%%%%%%%%%%%%%%%%%%%%%%%%%
% PDF Meta information                                 				       %
%%%%%%%%%%%%%%%%%%%%%%%%%%%%%%%%%%%%%%%%%%%%%%%%%%%%%%%%%%%%%%%%%%%%%%
\hypersetup{
  pdfauthor   = {\authorName},
  pdfkeywords = {\tags},
  pdftitle    = {\projektName~(Pflichtenheft)}
} 

%%%%%%%%%%%%%%%%%%%%%%%%%%%%%%%%%%%%%%%%%%%%%%%%%%%%%%%%%%%%%%%%%%%%%%
% Custom setup                                                       % 
%%%%%%%%%%%%%%%%%%%%%%%%%%%%%%%%%%%%%%%%%%%%%%%%%%%%%%%%%%%%%%%%%%%%%%
\usepackage{pdfpages}
\usepackage{xcolor,colortbl} 
\definecolor{Blue}{rgb}{0.1,0.2,0.7}
\definecolor{TUMBlue}{HTML}{0065BD}
\definecolor{TUMSecondaryBlue}{HTML}{005293}
\definecolor{TUMSecondaryBlue2}{HTML}{003359}
\definecolor{TUMAccentBlue}{HTML}{64A0C8}
\definecolor{TUMAccentLightBlue}{HTML}{98C6EA}

\newcolumntype{a}{>{\columncolor{TUMBlue}}c}
\renewcommand{\arraystretch}{1.5}
 
%%%%%%%%%%%%%%%%%%%%%%%%%%%%%%%%%%%%%%%%%%%%%%%%%%%%%%%%%%%%%%%%%%%%%%
% Create a shorter version for tables. DO NOT CHANGE               	 %
%%%%%%%%%%%%%%%%%%%%%%%%%%%%%%%%%%%%%%%%%%%%%%%%%%%%%%%%%%%%%%%%%%%%%%
\newcommand\addrow[2]{\textcolor{white}{#1} &#2\\ \hline}

\newcommand\addheading[2]{\rowcolor{TUMBlue}\textcolor{white}{#1} & \textcolor{white}{#2}\\ \hline}
\newcommand\tabularhead{\begin{tabular}{|a|p{13cm}|}
\hline
}

\newcommand\addmulrow[2]{ \begin{minipage}[t][][t]{2.5cm}#1\end{minipage}% 
   &\begin{minipage}[t][][t]{8cm}
    \begin{enumerate} #2   \end{enumerate}
    \end{minipage}\\ }

\newenvironment{usecase}{\tabularhead}
{\hline\end{tabular}}

%%%%%%%%%%%%%%%%%%%%%%%%%%%%%%%%%%%%%%%%%%%%%%%%%%%%%%%%%%%%%%%%%%%%%%
% THE DOCUMENT BEGINS             	                              	 %
%%%%%%%%%%%%%%%%%%%%%%%%%%%%%%%%%%%%%%%%%%%%%%%%%%%%%%%%%%%%%%%%%%%%%%
\begin{document}
 \pagenumbering{roman}
 \begin{titlepage}
\maketitle
\thispagestyle{empty} % no page number

\begin{verbatim}












\end{verbatim}


  \begin{tabular}[t]{ll}
	Projekt:       & \quad \projektName \\[1.2ex]
	Auftraggeber:  & \quad \auftraggeber\\[1.2ex]
	Auftragnehmer: & \quad \auftragnehmer\\[1.2ex]
  \end{tabular}

\begin{tabular}{|p{3 cm}|p{3 cm}|p{5 cm}|}
\hline
\textbf{Version} & \textbf{Datum} & \textbf{Autor(en)} \\
\hline
\hline
1.0 & 04.11.2019 & \authorName \\
\hline
\end{tabular}
\end{titlepage}
         % Deckblatt.tex laden und einfügen
 \setcounter{page}{2}

 \tableofcontents          % Inhaltsverzeichnis ausgeben
 \clearpage
 \pagenumbering{arabic}
 
\section{Zielbestimmung}
%%%%%%%%%%%%%%%%%%%%%%%%%%%%%%%%%%%%%%%%%%%%%%%%%%%%%%%%%%%%%%%%%%%%%%
% Warum wird das Projekt gemacht?           						 %
%%%%%%%%%%%%%%%%%%%%%%%%%%%%%%%%%%%%%%%%%%%%%%%%%%%%%%%%%%%%%%%%%%%%%%
Der Freistaat Bayern möchte in Kooperation mit dem Lehrstuhl für 
Datenbanksysteme an der Technischen Universität München eine digitales 
Wahlinformations- und Stimmabgabesystem für Landtagswahlen mithilfe von 
Studierenden des Elite Software Engineering Masterstudiengangs aufbauen.
%
Das System soll dabei nicht nur die Ergebnisse für die Landtagswahlen 
2013 und 2018 analysier- und vergleichbar machen, e.g., die Sitzverteilung 
im Landtag, statistische Auswertung von Ergebnissen, Berechnung gewonnener
Mandate, sondern auch als sicheres Backendsystem für die elektronische 
Stimmabgabe im Wahllokal dienen. 
%
Es müssen die gesetzlichen Regelungen, Normen und Datenschutzaspekte
berücksichtigt werden. Als Basis gelten die Regelungen zur
Landtagswahl 2018


\subsection{Musskriterien}
% Requirements
\subsection{Sollkriterien}
\subsection{Kannkriterien}
\subsection{Abgrenzungskriterien}

\section{Technische Umsetzung}
\subsection{Produkteinsatz und Umgebung}
\subsection{Entwicklungsumgebung}

\section{GUI Mockups}

\section{Datenmodell}

\section{Globale Testfälle und Szenarien}
% 10 Leute gleichzeitig in Wahllokal greifen zu
% Auswertung realer wahlergebnisse geht flott

\clearpage
%%%%%%%%%%%%%%%%%%%%%%%%%%%%%%%%%%%%%%%%%%%%%%%%%%%%%%%%%%%%%%%%%%%%%%
% Begriffslexikon zur Beschreibung des Produkts						 %
%%%%%%%%%%%%%%%%%%%%%%%%%%%%%%%%%%%%%%%%%%%%%%%%%%%%%%%%%%%%%%%%%%%%%%
%\newglossaryentry{sortierschluessel}
%{
%  name=Sortierschlüssel,
%  description={ein Schlüssel, anhand dessen diese Einträge sortiert werden}
%}
\newglossaryentry{Technologiestack}
{
  name=Technologiestack,
  description={Sammlung aller verwendeten Technologien im Projekt. Gegebenenfalls hierarchisch sortiert wenn aufeinander aufbauend}
}
\newglossaryentry{Client}
{
  name=Client,
  description={Programm, dass die Dienste eines Servers in Anspruch nimmt}
}
\newglossaryentry{Server}
{
  name=Server,
  description={Rechner, der für andere in einem Netzwerk mit ihm verbundene Systeme bestimmte Aufgaben übernimmt und von dem diese ganz oder teilweise abhängig sind}
}
\newglossaryentry{Datenbanksystem}
{
  name=Datenbanksystem,
  description={Ein Datenbanksystem (DBS) ist eine systematisch strukturierte, langfristig verfügbare Sammlung von Daten einschließlich der zur Verwaltung notwendigen Software}
}
\newglossaryentry{OLAP}
{
  name=OLAP,
  description={Online Analytical Processing}
}
\newglossaryentry{OLTP}
{
  name=OLTP,
  description={Online Transaction Processing}
}


% Setze den richtigen Namen für das Glossar
\renewcommand*{\glossaryname}{\section{\glossarName}}

% Drucke das gesamte Glossar
\glsaddall
\printglossaries

% Trage das Glossar in das Inhaltsverzeichnis ein
\stepcounter{section}
\addcontentsline{toc}{section}{\numberline {\thesection} \glossarName}
\end{document}
