\documentclass[a4paper,12pt]{article}
%spellchecker
% !TeX spellcheck = de

\usepackage{amssymb} % needed for math
\usepackage{amsmath} % needed for math
\usepackage[utf8]{inputenc} % this is needed for german umlauts
\usepackage[ngerman]{babel} % this is needed for german umlauts
\usepackage[T1]{fontenc}    % this is needed for correct output of umlauts in pdf
\usepackage[margin=2.5cm]{geometry} %layout
\usepackage{booktabs}

% this is needed for forms and links within the text
\usepackage{hyperref}  

% Generate the glossary
\usepackage[nonumberlist]{glossaries}
\makeglossary

% The following is needed in order to make the code compatible
% with both latex/dvips and pdflatex.
\ifx\pdftexversion\undefined
\usepackage[dvips]{graphicx}
\else
\usepackage[pdftex]{graphicx}
\DeclareGraphicsRule{*}{mps}{*}{}
\fi

%%%%%%%%%%%%%%%%%%%%%%%%%%%%%%%%%%%%%%%%%%%%%%%%%%%%%%%%%%%%%%%%%%%%%%
% Variablen                                 						 %
%%%%%%%%%%%%%%%%%%%%%%%%%%%%%%%%%%%%%%%%%%%%%%%%%%%%%%%%%%%%%%%%%%%%%%
\newcommand{\authorName}{Lena Gregor, Dominik Horn}
\newcommand{\auftraggeber}{Lehrstuhl für Datenbanksysteme TUM}
\newcommand{\auftragnehmer}{\authorName}
\newcommand{\projektName}{Wahl und Informationssystem für bayrische Landtagswahlen}
\newcommand{\tags}{\authorName, Pflichtenheft, TUM, Universität Augsburg, LMU}
\newcommand{\subtitle}{Technische Universität München, Datenbanksysteme WS19/20}
\newcommand{\glossarName}{Glossar}

%%%%%%%%%%%%%%%%%%%%%%%%%%%%%%%%%%%%%%%%%%%%%%%%%%%%%%%%%%%%%%%%%%%%%%
% PDF Meta information                                 				       %
%%%%%%%%%%%%%%%%%%%%%%%%%%%%%%%%%%%%%%%%%%%%%%%%%%%%%%%%%%%%%%%%%%%%%%
\hypersetup{
  pdfauthor   = {\authorName},
  pdfkeywords = {\tags},
  pdftitle    = {\projektName~(Pflichtenheft)}
} 

%%%%%%%%%%%%%%%%%%%%%%%%%%%%%%%%%%%%%%%%%%%%%%%%%%%%%%%%%%%%%%%%%%%%%%
% Custom setup                                                       % 
%%%%%%%%%%%%%%%%%%%%%%%%%%%%%%%%%%%%%%%%%%%%%%%%%%%%%%%%%%%%%%%%%%%%%%
\usepackage{pdfpages}
\usepackage{xcolor,colortbl} 
\definecolor{Blue}{rgb}{0.1,0.2,0.7}
\newcolumntype{a}{>{\columncolor{Blue}}c}
\renewcommand{\arraystretch}{1.5}
 
%%%%%%%%%%%%%%%%%%%%%%%%%%%%%%%%%%%%%%%%%%%%%%%%%%%%%%%%%%%%%%%%%%%%%%
% Create a shorter version for tables. DO NOT CHANGE               	 %
%%%%%%%%%%%%%%%%%%%%%%%%%%%%%%%%%%%%%%%%%%%%%%%%%%%%%%%%%%%%%%%%%%%%%%
\newcommand\addrow[2]{\textcolor{white}{#1} &#2\\ \hline}

\newcommand\addheading[2]{\rowcolor{Blue}\textcolor{white}{#1} & \textcolor{white}{#2}\\ \hline}
\newcommand\tabularhead{\begin{tabular}{|a|p{13cm}|}
\hline
}

\newcommand\addmulrow[2]{ \begin{minipage}[t][][t]{2.5cm}#1\end{minipage}% 
   &\begin{minipage}[t][][t]{8cm}
    \begin{enumerate} #2   \end{enumerate}
    \end{minipage}\\ }

\newenvironment{usecase}{\tabularhead}
{\hline\end{tabular}}

%%%%%%%%%%%%%%%%%%%%%%%%%%%%%%%%%%%%%%%%%%%%%%%%%%%%%%%%%%%%%%%%%%%%%%
% THE DOCUMENT BEGINS             	                              	 %
%%%%%%%%%%%%%%%%%%%%%%%%%%%%%%%%%%%%%%%%%%%%%%%%%%%%%%%%%%%%%%%%%%%%%%
\begin{document}
 \pagenumbering{roman}
 \begin{titlepage}
\thispagestyle{empty} % no page number
\centering

\vspace*{4cm}
{\huge\bfseries Pflichtenheft\par}
\vspace{1cm}
{\huge\bfseries \projektName\par}
\vspace{0.5cm}
{\large \subtitle\par}
\vspace{1cm}
{\large\itshape \authorName\par}

\vfill

%\renewcommand{\arraystretch}{1.5}
\begin{tabular}{|a|l|}
	\hline
	\textcolor{white}{\textbf{Projekt}} & \projektName \\
	\hline
	\textcolor{white}{\textbf{Auftraggeber}} & \auftraggeber \\
	\hline
	\textcolor{white}{\textbf{Auftragnehmer}} & \auftragnehmer \\
	\hline
	\textcolor{white}{\textbf{Datum}} & \today \\
	\hline
\end{tabular}
%\renewcommand{\arraystretch}{1}

\end{titlepage}
         % Deckblatt.tex laden und einfügen
 \setcounter{page}{2}
 
 % TODO: TEMPORARY
 THIS DOCUMENT IS A COPY OF LASTENHEFT AND NEEDS ACTUAL CONTENT FILLED IN
 \clearpage

 \tableofcontents          % Inhaltsverzeichnis ausgeben
 \clearpage
 \pagenumbering{arabic}
 
\section{Zielbestimmung}
%%%%%%%%%%%%%%%%%%%%%%%%%%%%%%%%%%%%%%%%%%%%%%%%%%%%%%%%%%%%%%%%%%%%%%
% Warum wird das Projekt gemacht?           						 %
%%%%%%%%%%%%%%%%%%%%%%%%%%%%%%%%%%%%%%%%%%%%%%%%%%%%%%%%%%%%%%%%%%%%%%
Der Freistaat Bayern möchte in Kooperation mit dem Lehrstuhl für 
Datenbanksysteme an der Technischen Universität München eine digitales 
Wahlinformations- und Stimmabgabesystem für Landtagswahlen mithilfe von 
Studierenden des Elite Software Engineering Masterstudiengangs aufbauen.
%
Das System soll dabei nicht nur die Ergebnisse für die Landtagswahlen 
2013 und 2018 analysier- und vergleichbar machen, e.g., die Sitzverteilung 
im Landtag, statistische Auswertung von Ergebnissen, Berechnung gewonnener
Mandate, sondern auch als sicheres Backendsystem für die elektronische 
Stimmabgabe im Wahllokal dienen. 
%
Es müssen die gesetzlichen Regelungen, Normen und Datenschutzaspekte
berücksichtigt werden. Als Basis gelten die Regelungen zur
Landtagswahl 2018

\section{Produkteinsatz}
%%%%%%%%%%%%%%%%%%%%%%%%%%%%%%%%%%%%%%%%%%%%%%%%%%%%%%%%%%%%%%%%%%%%%%
% Wer ist die Zielgruppe?                   						 %
%%%%%%%%%%%%%%%%%%%%%%%%%%%%%%%%%%%%%%%%%%%%%%%%%%%%%%%%%%%%%%%%%%%%%%
Zielgruppe ist im Wesentlichen der Übungsleiter der Database Systems 
Vorlesung (Christian Winter) und die Studierenden/Entwickler. 
%
Konzipiert und Entwickelt werden soll die Anwendung jedoch zur Verwendung 
durch Wahlhelfer zur Stimmeneintragung für Bayrische Landtagswahlen, 
zur Stimmabgabe durch Wahlberechtigte und zur (statistischen)-Auswertung
von Wahlergebnissen durch den oder die WahlleiterIn sowie u.U. 
interessierten BürgerInnen.

\begin{center}
\begin{tabular}{|m{5cm}|m{10cm}|}
	\hline
  \rowcolor{Blue} \textcolor{white}{\textbf{Einsatzgebiet}} & \textcolor{white}{\textbf{Prozesse}} \\
  \hline
  Wahllokal & Abgabe von Einzelstimmen durch WählerInnen und batch Stimmeintragung durch WahlhelferInnen \\
	\hline
  Bürgerlicher Gebrauch & (statistische-)Analyse von Wahlergebnissen \\
  \hline
  Staatlicher Gebrauch & (statistische-)Analyse von Wahlergebnissen, Import von alten Wahlergebnissen \\
	\hline
\end{tabular}
\end{center}


\section{Benutzerschnittstellen}
Dieser Absatz beschäftigt sich mit der Definition von Benutzerschnittstellen und fundamentalen Usecases:

\begin{enumerate}
  \item Die Informationsabfrage und (statistische-)Auswertung von Wahlergebnissen erfolgt
        in einer frei zugänglichen Weboberfläche. Hierbei soll auch der Vergleich zwischen 
        verschiedenen Wahlen möglich sein. 
  \item Das System soll geeignet sein für die elektronische Stimmabgabe im Wahllokal durch
        wahlberechtigte Personen
  \item Um zu gewährleisten, dass jeweils nur eine Stimme pro wahlberechtigter Person abgegeben
        werden kann, muss die Stimmabgabe erst freigegeben werden. Dies geschieht entweder nach
        Authentifizierung durch eine(n) WahlhelferIn oder durch den oder die WählerIn selbst 
        (Unique one-time-use password, Vorlage von Personalausweis/Reisepass).
  \item WahlhelferInnen müssen in der Lage sein ein oder mehrere Stimmen auf einmal ins System
        eintragen zu können, zum Beispiel um analoge Stimmen ins System zu übernehmen. 
  \item Insbesondere zum Vergleich von (statistischen-)Auswertungen verschiedener Wahlen können
        alte Wahldaten und Ergebnisse von der oder dem WahlleiterIn ins System aus CSV-Dateien 
        importiert werden.
  \item Der oder die WahlleiterIn muss imstande sein Stimmen im System zu korrigieren, zum Beispiel
        nach einer Neuauszählung von Stimmen
\end{enumerate}

\begin{center}
	\includegraphics[width=\textwidth]{../usecases.pdf}
\end{center}



\section{Funktionale Anforderungen}
%%%%%%%%%%%%%%%%%%%%%%%%%%%%%%%%%%%%%%%%%%%%%%%%%%%%%%%%%%%%%%%%%%%%%%
% Was muss das Programm können?                   					 %
%%%%%%%%%%%%%%%%%%%%%%%%%%%%%%%%%%%%%%%%%%%%%%%%%%%%%%%%%%%%%%%%%%%%%%
\begin{usecase}
  \addheading{Nummer}{Beschreibung} 
  \addrow{FR01}{Wahldaten müssen über eine graphische Schnittstelle (statistisch) auswertbar sein. Hierbei ist insbesondere das Bestimmen von
                Wahlergebnissen, d.h. gewonnenen Mandaten, wichtig. Wahldaten unterschiedlicher Wahlvorgänge können dabei verglichen werden}
  \addrow{FR02}{Einzelstimmen sollen durch Wahlberechtigte im Wahllokal abgegeben werden können (E-Voting). Die notwendige Authentifizierung
                erfolgt dabei zum Beispiel durch eine(n) WahlhelferIn}
  \addrow{FR03}{Die Einzelstimmabgabe muss vor Benutzung durch eine wahlberechtigte Person je freigegeben werden. 
                Dies kann zum Beispiel durch eine(n) WahlhelferIn geschehen nach analoger Verifikation von Reisepass oder Personalausweis}
  \addrow{FR04}{WahlhelferInnen müssen in der Lage sein mehrerer Stimmen auf einmal (batch) in das System einzutragen. Dies ist zum Beispiel
                dienlich um analoge Stimmabgaben in das System zu übertragen}
  \addrow{FR05}{Wahldaten von Landtagswahlen ab (inklusive) 2013 müssen fehlerfrei aus einer CSV-Datei importierbar und im System abbildbar sein.}
  \addrow{FR06}{WahlleiterInnen haben die Möglichkeit Stimmen zu korrigieren, zum Beispiel falls die Gültigkeit von Stimmen nachträglich annuliert 
                werden muss oder es im Rahmen von Neuauszählungen zu Änderungen kommt.}
  \addrow{FR07}{Pro Stimmkreis werden die Einzelstimmen zu Stimmkreisergebnissen voraggregiert.}
\end{usecase}

\section{Nichtfunktionale Anforderungen}
\begin{usecase}
  \addheading{Nummer}{Beschreibung} 
  \addrow{NF01}{Das System muss alle rechtlichen Vorgaben zur Landtagswahl im Freistaat Bayern einhalten, insbesondere im Bezug auf Datenschutz und Wahlgeheimnis.
                Für letzteres ist beispielsweise das getrennte abspeichern von Erst- und Zweitstimmen fundamental.
                Juristische Ausnahmefälle, die bei den Wahlen 2013 und 2018 nicht relevant waren und über 
                Sperrklausel, Überhangsmandate, Ausgleichsmandate hinausgehen, müssen im System nicht verarbeitet/ berechnet werden können.}
  \addrow{NF02}{Zu Peak-Lastzeiten, i.e., wenn tausende WählerInnen gleichzeitig abstimmen, muss das System reibungslos einsatzfähig bleiben.}
  \addrow{NF03}{Ein wirkungsvoller Schutz vor unbefugten Zugriffen muss ein fester Bestandteil sein um die Integrität der Wahlergebnisse nicht zu gefährden.}
  \addrow{NF04}{Die (statistische-)Auswertung von Wahlergebnissen muss schnell und korrekt sein.}
  \addrow{NF05}{Die graphische Oberfläche soll mächtig, jedoch intuitiv sein.}
\end{usecase}

\section{Produktdaten}
Die grundlegenden zu verwaltenden Produktdaten ergeben sich aus der folgenden Liste von
identifizierten Entitäten:
\begin{itemize}
  \item \textbf{Wahl} - Repräsentation einer einzelnen Wahl an einem festen Datum. 
        Durch letzteres Merkmal ist es möglich mehrere Wahlen in einem Jahr festzuhalten.
  \item \textbf{Stimme} - Jede(r) WählerIn besitzt je eine Erst- und Zweitstimme, welche er oder sie in Ihrem
        Stimmbezirk für KandidatInnen abgeben können. Mit der Erststimme wählt eine wahlberechtigte Person
        primär den oder die von ihr bevorzugten DirektkandidatIn ihres Stimmkreises. Über die Zweitstimme wählt 
        jede wahlberechtigte Person genau einen Kandidaten einer Liste aus ihrem Regierungsbezirk. Das prozentuale
        Gesamtabschneiden einer Partei aus kombinierten Erst- und Zweitstimmen ist jedoch ebenso für das 
        Wahlergebnis relevant.
  \item \textbf{Regierungsbezirk} - Bayern ist in sieben Regierungsbezirke unterteilt: Oberbayern,
        Niederbayern, Schwaben, Ober-, Unter-, Mittelfranken und Oberpfalz. Jeder Regierungsbezirk
        hat eine unterschiedliche Anzahl zu vergebender Direkt- und Listenmandate
  \item \textbf{Stimmkreis} - Regierungsbezirke sind gemäß Bevölkerungszahlen in Stimmkreise aufgeteilt.
        Jeder Stimmkreis ist widerum in ein oder mehrere Stimmbezirke unterteilt. Pro Stimmkreis wird ein(e)
        DirektkandidatIn gewählt
  \item \textbf{Stimmbezirk} - Die Stimmabgabe erfolgt in Stimmbezirken
  \item \textbf{KandidatIn} - Eine sich zur Wahl stellende Person mit notwendigen persönlichen Angaben. 
        Insbesondere kann ein(e) KandidatIn als DirektkandidatIn für einen Stimmkreis aufgestellt sein.
  \item \textbf{Mandat} - Mandate für den Bayrischen Landtag können von KandidatInnen im Rahmen einer Landtagswahl
        errungen werden. Hierbei wird zwischen Direktmandaten, Listenmandaten und Ausgleichsmandaten unterschieden.
  \item \textbf{Partei} - KandidatInnen können Mitglied einer Partei sein und sich von dieser für einen Listenplatz
        aufstellen lassen
  \item \textbf{Liste} - Jede Partei kann pro Regierungsbezirk eine Liste mit KandidatInnen, welche Mitglieder der 
        Partei sind, aufstellen.
\end{itemize}

\begin{center}
	\includegraphics[width=\textwidth]{../model.pdf}
\end{center}

\section{Abnahmekriterien}
Dieser Abschnitt definiert die wichtigsten Kriterien welche zwischen dem Erzeugnis und der Abnahme stehen.
Sie leiten sich hauptsächlich aus den in diesem Dokument spezifizierten Anforderungen ab.

\begin{enumerate}
  \item Die Softwareplatform ist über eine graphisch ansprechende Weboberfläche benutzbar.
  \item Das System erlaubt die (statistische) Auswertung von realen Stimmdaten zu bayrischen Landtagswahlen, d.h. 
        insbesondere die Bestimmung der gewonnenen Mandate im bayrischen Landtag. Dabei müssen gesetzliche
        Vorgaben genauestens eingehalten sein sowie Daten sicher, Datenschutzrechtlich korrekt und integer gehalten
        werden.
  \item WählerInnen haben durch eine bereitgestellte Oberfläche die Möglichkeit ihre Erst- und Zweitstimme abzugeben.
        Dabei muss gewährleistet sein, dass jede Wahlberechtigte Person nur maximal einmal abstimmen kann, 
        Datenschutzrechtliche sowie gesetzliche Vorgaben wie das Stimmgeheimnis eingehalten werden und das System 
        insbesondere vor unbefugtem Zugriff geschützt ist.
  \item WahlhelferInnen können schnell Stimmen, e.g. analog abgegebene, in einer minimalen Anzahl von Schritten in das
        System übernehmen.
  \item Reale Stimmdaten vergangener Wahlen können von berechtigten Nutzern, e.g., WahlleiterInnen, importiert und
        (statistisch) Ausgewertet werden. Hierbei ist insbesondere auch der vergleich mit den Ergebnissen anderer
        Wahlvorgänge wichtig.
\end{enumerate}

\clearpage
%%%%%%%%%%%%%%%%%%%%%%%%%%%%%%%%%%%%%%%%%%%%%%%%%%%%%%%%%%%%%%%%%%%%%%
% Begriffslexikon zur Beschreibung des Produkts						 %
%%%%%%%%%%%%%%%%%%%%%%%%%%%%%%%%%%%%%%%%%%%%%%%%%%%%%%%%%%%%%%%%%%%%%%
%\newglossaryentry{sortierschluessel}
%{
%  name=Sortierschlüssel,
%  description={ein Schlüssel, anhand dessen diese Einträge sortiert werden}
%}
\newglossaryentry{Technologiestack}
{
  name=Technologiestack,
  description={Sammlung aller verwendeten Technologien im Projekt. Gegebenenfalls hierarchisch sortiert wenn aufeinander aufbauend}
}
\newglossaryentry{Client}
{
  name=Client,
  description={Programm, dass die Dienste eines Servers in Anspruch nimmt}
}
\newglossaryentry{Server}
{
  name=Server,
  description={Rechner, der für andere in einem Netzwerk mit ihm verbundene Systeme bestimmte Aufgaben übernimmt und von dem diese ganz oder teilweise abhängig sind}
}
\newglossaryentry{Datenbanksystem}
{
  name=Datenbanksystem,
  description={Ein Datenbanksystem (DBS) ist eine systematisch strukturierte, langfristig verfügbare Sammlung von Daten einschließlich der zur Verwaltung notwendigen Software}
}
\newglossaryentry{OLAP}
{
  name=OLAP,
  description={Online Analytical Processing}
}
\newglossaryentry{OLTP}
{
  name=OLTP,
  description={Online Transaction Processing}
}


% Setze den richtigen Namen für das Glossar
\renewcommand*{\glossaryname}{\section{\glossarName}}

% Drucke das gesamte Glossar
\glsaddall
\printglossaries

% Trage das Glossar in das Inhaltsverzeichnis ein
\stepcounter{section}
\addcontentsline{toc}{section}{\numberline {\thesection} \glossarName}
\end{document}
